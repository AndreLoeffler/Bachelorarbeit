\begin{chapter}{Einleitung}
 Die Formulierung des 4-Farben Satzes geht auf eine Beobachtung zurück, die Francis Guthrie 1852 machte. Francis Guthrie war gelernter Jurist, Hobbybotaniker und Mathematiker. Als er versuchte, eine Landkarte der Grafschaften Englands zu illustrieren und kam zu einer recht anschaulichen Vermutung. \\
 Francis' Bruder Frederick Guthrie wand sich damit Problem an seinen Lehrer Augustus de Morgan. Fasziniert von dieser Problematik schrieb de Morgan einen Brief an Sir William Rowan Hamilton. Dieser Notiz ist die erste schriftliche Formulierung des Vierfarbenproblems zu entnehmen:
 
 \begin{satz}[historische Formulierung]
  A student of mine asked me to day to give him reason for a fact which I did not know was a fact, and do not yet. He says, that if a figure be any how divided and the compartments differently coloured so that figures with any portion of common boundary \underline{line} are differently coloured -- four colours may be wanted but not more. The following is his care in which four \underline{are} wanted. [...]\\
  Query cannot a necessity for five or more be invented. As far as I see at this moment, if four \underline{ultimate} compartments have each boundary line in common with one of the others, three of them inclose the fourth, and prevent any fifth from connexion with it. If this be true, four colours will colour any possible map without any necessity for colour meeting colour except at a point.  \cite{fritsch}
 \end{satz}
 
 Die ursprüngliche Fragestellung lautet also: Kann man eine beliebige Landkarte so einfärben, dass keine zwei benachbarten Länder die gleiche Farbe haben, wenn man die Farbpalette auf vier Farben beschränkt?
 
 Um eine Landkarte als mathematisches Konstrukt auffassen zu können, bedarf es einiger topologischer Hilfsmittel. 

\end{chapter}