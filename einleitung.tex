\begin{chapter}{Einleitung}
 Die Formulierung des Vier-Farben Satzes geht auf eine Beobachtung zurück, die Francis Guthrie im Jahr 1852 machte. Francis Guthrie studierte Rechtswissenschaften, war Hobbybotaniker und hatte einen Abschluss als Mathematiker. Er versuchte, eine Landkarte der Grafschaften Englands zu illustrieren, und kam zu einer recht anschaulichen Vermutung, die Mathematiker 150 Jahre lang beschäftigen sollte. \\
 Francis' Bruder Frederick Guthrie wand sich am 23. Oktober 1852 mit diesem Problem an seinen Lehrer Augustus de Morgan, der zu dieser Zeit am University College in London unterrichtet.\footnote{Um alle beteiligten Personen, ihre Lebensläufe und ihr Zusammenwirken besser kennenzulernen empfiehlt sich die Lektüre des ersten Kapitels von \cite{fritsch}} Fasziniert von dieser Problematik schrieb de Morgan einen Brief an Sir William Rowan Hamilton. Dieser Notiz ist die erste schriftliche Formulierung des Vierfarbenproblems zu entnehmen:
 
 \begin{satz}{historische Formulierung}
  A student of mine asked me to day to give him reason for a fact which I did not know was a fact, and do not yet. He says, that if a figure be any how divided and the compartments differently coloured so that figures with any portion of common boundary \underline{line} are differently coloured -- four colours may be wanted but not more. The following is his care in which four \underline{are} wanted. [...]\\
  Query cannot a necessity for five or more be invented. As far as I see at this moment, if four \underline{ultimate} compartments have each boundary line in common with one of the others, three of them inclose the fourth, and prevent any fifth from connexion with it. If this be true, four colours will colour any possible map without any necessity for colour meeting colour except at a point. \cite{fritsch}
 \end{satz}
 
 Die ursprüngliche Fragestellung lautet also: Kann man eine beliebige Landkarte so einfärben, dass keine zwei Länder, die sich den Abschnitt einer Grenzlinie teilen, die gleiche Farbe haben, wenn man die Farbpalette auf vier Farben beschränkt? Eine Landkarte lässt sich als mathematisches Konstrukt auffassen, jedoch bedarf es dazu einiger Überlegungen. R. und G. Fritsch definieren eine Landkarte $\mathcal{L}$ als ``[...] eine endliche Menge von Jordanbögen in der Ebene $\mathbb{R}^2$ derart, dass der Durchschnitt von je zwei verschiedenen Jordanbögen in $\mathcal{L}$ entweder leer oder ein gemeinsamer Randpunkt dieser Jordanbögen ist.'' \cite{fritsch} Diese Definition erscheint auf den ersten Blick eigenartig, benutzt sie doch keinen der zu erwartenden Begriffe wie ``Land'' oder ``Grenze''. 
 
 Die historische Formulierung wirkt nach heutigen Maßstäben etwas geschwollen und ist sprachlich nicht mehr zeitgemäß. Heute werden Aussagen zumeist prägnanter abgefasst. Bei \cite{fritsch} findet man eine aktuelle Variante auf Seite 87:  
 
 \begin{satzl}{topologische Formulierung}{topo}
  Es seien $\mathcal{L}$ eine Landkarte, $\mathcal{M}_\mathcal{L}$ die Menge der Länder von $\mathcal{L}$ und $n \in \mathbb{N}$. Eine \textit{$n$-Färbung} von $\mathcal{L}$ ist eine Abbildung $\varphi: \mathcal{M}_{\mathcal{L}} \rightarrow \{1,\cdots,n\}$. Eine $n$-Färbung ist \textit{zulässig}, wenn benachbarte Länder immer verschiedene Werte (``Farben'') haben.
 \end{satzl}

 Um dies korrekt erfassen und schließlich auch beweisen zu können, werden topologische Resultate wie der Jordansche Kurvensatz benutzt. Diese wiederum erfordern zahlreiche Vorüberlegungen, die sich sehr umfangreich gestalten und wenig zum eigentlichen Beweis beitragen. Stattdessen werden wir eine andere Formulierung des Vier-Farben-Satzes benutzen, die kombinatorisch motiviert ist. 
 
 \begin{satzl}{Graphentheoretische Formulierung}{graph}
  Jeder planare Graph ohne Schleifen ist 4-färbbar.
 \end{satzl}
 
 Diese Variante wirft einige Fragen nach Begrifflichkeiten auf, welche jedoch bei genauerer Betrachtung leicht verständlich sind. Im nächsten Abschnitt werden wir uns zunächst den allgemeinen Definitionen widmen, die nötig sind, um diese Problematik graphentheoretisch angehen zu können. Danach zeigen wir die Äquivalenz von \ref{topo} und \ref{graph} und beschreiben die Resultate und Überlegungen, welche beiden hier vorgestellten Beweisen zugrunde liegen. Im Anschluss erläutern wir die Arbeit von Robertson, Sanders, Seymour und Thomas in den wesentlichen Schritten ``Reduktion'' und ``Zwangsläufigkeit''. Im Anschluss vergleichen wir den neueren Beweis mit dem Vorgänger von Appel \& Haken, wobei wir uns auf die Unterschiede konzentrieren wollen. Abschließend werden noch einige Umformulierungen und Anwendungen des Vier-Farben-Satzes diskutiert.
\end{chapter}