\begin{section}{Kombinatorische Definitionen}
  Um über Graphen und deren Färbbarkeit sinnvoll reden zu können, müssen zuerst einige gebräuchliche Definitionen gemacht werden. 
 
  \begin{definition}{endlicher Graph, Knoten, Kante}
    \-\ 
    \begin{itemize}
    \item Ein \textit{endlicher Graph} $G$ ist ein Tupel $G=(V,E)$, wobei $V$ eine endliche Menge bestehend aus Knoten und $E$ eine endliche Menge bestehend aus Kanten sind.
    \item Ein \textit{Knoten} $v \in V$ ist ein Punkt im Raum. 
    \item Eine \textit{Kante} $e \in E$ ist eine zweielementige Teilmenge von $V$, wobei $E$ die Menge aller dieser Teilmengen ist, also $E = \{\{u,v\}|u,v \in V\}$.
    \end{itemize}
  \end{definition}
  
  Oft werden wir auch die Bausteine einer Landkarte $\mathcal{L}$ als Knoten- und Kantenmenge verwenden. Die Ecken von $\mathcal{L}$ werden zu den Knoten des Graphen, die Kanten von $\mathcal{L}$ zu denen des Graphen. Im weiteren betrachten wir vorwiegend endliche Graphen, außer es wird explizit anders angegeben.\\
  Um nun Bedingungen an die Färbbarkeit von Knoten stellen zu können, muss noch definiert werden, wie diese zusammenhängen.
 
  \begin{definition}{Inzidenz, Adjazenz, Knotengrad, vollständiger Graph}
    \-\ 
    \begin{itemize}
    \item Ein Knoten $v \in V$ heißt \textit{inzident} zu einer Kante $e \in E$, wenn mindestens einer der Endpunkte von $e$ der Knoten $v$ ist. 
    \item Zwei Knoten $u,v$ heißen \textit{adjazent}, wenn sie zur gleichen Kante inzident sind. 
    \item Für einen Knoten $v$ ist der \textit{Grad} von $v$ definiert als die Anzahl der Kanten, die zu $v$ inzident sind. Es gilt $d_G(v) = \sharp\{\{a,b\} \in E | a=v \wedge b=v \}$.
    \item Ein Graph $G$ heißt \textit{vollständig}, wenn keine Kante mehr hinzugenommen werden kann, ohne das die Knotenmenge erweitert werden muss.
   \end{itemize}
  \end{definition}
    
  Für eine interessante Struktur benachbarter Knoten gibt es eine gebräuchliche Bezeichnung, auf die wir später zurückgreifen werden.
  
  \begin{definition}{Pfad, einfacher Pfad, disjunkte Pfade, geschlossener Pfad, Ring}
  \-\ 
   \begin{itemize}
    \item Eine Folge $(e_1,\cdots,e_r)$ von mindestens drei Knoten heißt \textit{Pfad}, wenn die auftretenden Knoten paarweise verschieden, aber je zwei aufeinanderfolgenden benachbart sind. Dann ist $r$ die Länge der Kette und die Verbindungskanten heißen Glieder.
    \item Ein Pfad heißt \textit{einfach}, wenn zwei seiner Knoten nur dann benachbart sind, wenn sie im Pfad aufeinanderfolgenden. Oder genauer: \\
    Für die Indizes $j_1, j_2 \in \{1,\cdots,r\}$ mit $|j_1 - j_2| > 1$ sind die Ecken des Pfades $e_{j_1}$ und $e_{j_2}$ nicht benachbart.
    \item Zwei Pfade heißen \textit{disjunkt}, wenn sie keine inneren Knoten gemeinsam haben. 
    \item Ein Pfad heißt \textit{geschlossen}, wenn $e_1$ und $e_r$ ebenfalls benachbart sind.
    \item Ein \textit{Ring} ist ein Pfad, der sowohl einfach als auch geschlossen ist.
   \end{itemize}
  \end{definition}
  
  Ringe werden im Allgemeinen als wesentlicher Beitrag von Birkhoff auf dem Weg zur Lösung des 4-Farben-Problems angesehen, zuerst erwähnt in \cite{AmJMath35}.
  
  Da das Problem der 4-färbbarkeit von Graphen von der Geographie motiviert ist, betrachten wir als Raum für unsere Knoten nur den $\mathbb{R}^2$, also die Ebene.
  
  \begin{definition}{Planarität}
   Ein Graph heißt \textit{planar}, wenn er sich so in die Ebene einbetten lässt, dass sich zwei Kanten höchstens in ihrem gemeinsamen Endpunkt schneiden.
  \end{definition}
  
  \begin{definition}{Schleife}
   Eine Schleife ist eine Kante, deren beide Endpunkte der gleiche Knoten sind.
  \end{definition}
  Schleifen müssen bei Färbbarkeitsüberlegungen ausgeschlossen werden, denn könnte ein Knoten zu sich selbst benachbart sein, wäre es nicht möglich, für benachbarte Knoten stets unterschiedliche Farben zu wählen.
  
  In der Ebene ist es leicht, die durch die Kanten getrennten Flächen zu betrachten. Das führt uns zur folgenden Definitionen.
  
  \begin{definition}{Facette, Außenfacette}
   \-\ 
   \begin{itemize}
   \item Eine Fläche in der Ebene heißt \textit{Region} oder \textit{Facette}, falls sie vollständig von Kanten eingeschlossen ist. Die Endknoten der Kanten, die die Region umfassen heißen ebenfalls inzident zu dieser Region. 
   \item Der unbeschränkte Rest der Ebene, der von keiner Menge von Kanten vollständig umschlossen ist, wird \textit{Außenfacette} genannt.
   \end{itemize}
  \end{definition}
  
  Für unsere Betrachtungen ist eine besondere Form von Facetten interessant. 
  
  \begin{definition}{Dreieck, Triangulation, Beinahe-Triangulation}
   \-\ 
   \begin{itemize}
   \item Eine Region ist genau dann ein \textit{Dreieck}, wenn genau drei Knoten zu ihr inzident sind. Ein Ring ist genau dann ein Dreieck, wenn er aus drei Knoten besteht.
   \item Ein planarer Graph ist eine \textit{Triangulation}, wenn er schleifenfrei und jede seiner Facetten ein Dreieck ist. 
   \item Eine \textit{Beinahe-Triangulation} ist ein nichtleerer, schleifenfreier, planarer Graph $G$, bei dem jede endliche Facette ein Dreieck ist. 
   \end{itemize}
  \end{definition}
  
  Zeichnet man die Kanten eines planaren Graphen als gerade Linien, so entspricht diese Definition genau dem, was man sich unter einem Dreieck vorstellt. Das dies auch tatsächlich möglich ist, werden wir am Ende des Kapitels kurz diskutieren. Der Unterschied zwischen einer Triangulation und einer Beinahe-Triangulation liegt lediglich in der Form der Außenfacette des Graphen. 
 
  \begin{definition}{Färbung, Farben, Gültigkeit}
   \-\ 
   \begin{itemize}
   \item Eine \textit{Färbung} $f: V \rightarrow C \subset \mathbb{N}^0$ ist eine Abbildung, die jedem Knoten eines Graphen ein Element der endlichen Teilmenge $C = \{[0,n] \subset \mathbb{N}| n \in \mathbb{N}\}$ der natürlichen Zahlen zuordnet. 
   \item Die Elemente von $C$ nennt man \textit{Farben}. 
   \item Eine Färbung heißt \textit{gültig}, wenn sie keinem Paar adjazenter Knoten $u,v \in V$ die gleiche Farbe zuordnet, also $c(u) \neq c(v)$. 
   \end{itemize}
  \end{definition}
  
  \begin{definition}{$k$-Färbbarkeit}
   Ein Graph $G$ heißt \textit{$k$-färbbar}, wenn für eine gültige Färbung von $G$ höchstens $k$ Farben nötig sind. Insbesondere gilt dann: $\forall v \in V: f(v) < k$.
  \end{definition}
  
  Einiges Handwerkszeug ist noch nötig, um Strukturen prägnant und kurz beschreiben zu können.
  
  \begin{definition}{Teilgraph $G\setminus X$, $G\setminus Y$}
   Sei $G=(V,E)$ ein Graph, $X \subseteq V$ eine Teilmenge der Knoten und $Y \subseteq E$ eine Teilmenge der Kanten. Der Graph $G\setminus X = (E\setminus X,V)$ unterscheidet sich von $G$ derart, dass alle Knoten der Menge $X$ und alle zu diesen Knoten adjazenten Kanten gelöscht werden. Ebenso ist $G\setminus Y = (V,E\setminus Y)$ der Graph, bei dem alle Kanten aus $Y$ entfernt wurden.
  \end{definition}
  
  Nun haben wir alle nötigen Definitionen zusammen, um unsere ersten Resultate zu zeigen. Das erste dient vorallem der vereinfachten Veranschaulichung, das zweite werden wir später noch benötigen.
  
  \begin{satzl}{Der Satz von Wagner und Fáry}{WagFar}
   Jeder Graph kann durch einen Homöomorphismus der Ebene auf sich in einen Streckengraphen überführt werden.
  \end{satzl}
  
  Für den Beweis dieses Resultats verweisen wir auf \cite[Seite 113]{fritsch}, da er nur der Darstellung von Graphen nutzt und wenig zum eigentlichen Beweis beiträgt. Versucht man, einen Graphen darzustellen, werden die Kanten üblicherweise als Jordanbögen gezeichnet. Dieser Satz liefert uns, dass es sich bei diesen Bögen tatsächlich stets um gerade Verbindungsstrecken handeln kann.

  
  \begin{satzl}{Vollständiger Graph mit fünf Knoten}{voll5}
   Es existiert kein vollständiger planarer Graph mit fünf Knoten.
  \end{satzl}
  \begin{proof}
    Es seien $e_1,\cdots,e_5$ fünf Ecken in der Ebene. Für jedes Paar $i,j \in \{1,2,3,4,5\}$ mit $i < j$ sei eine Kante $k_{i,j}$ zwischen $e_i$ und $e_j$ gegeben. Dieser Graph hat $10$ Kanten, von denen insgesamt $7$ entweder $e_1$ oder $e_5$ als Endpunkt haben (oder beide). Wir zeigen nun, dass mindestens eine der $3$ übrigen Kanten eine der anderen Kanten schneiden muss.\\
    Durch Zusammensetzen erhält man drei Pfade $P_i = k_{1,i} \cup k_{i,5}$ für $i = 2,3,4$, die die Knoten $e_1$ und $e_5$ verbinden, die sich aber weder untereinander noch mit $k_{1,5}$ schneiden. O.B.d.A. sei $P_3$ der Pfad derart, dass von den beiden anderen einer in der Facette $F$ und der andere außerhalb der Facette $F$ begrenzt von $k_{1,5} \cup P_3$ liegt. Damit muss die Kante $k_{2,4}$ zwischen $e_2$ und $e_4$ mindestens einen inneren Punkt $y$ mit den Grenzen von $F$ gemeinsam haben, also mit einer der Kanten $k_{1,5},k_{1,3},k_{3,5}$. Da $k_{2,4}$ keine der drei beteiligten Ecken trifft, muss $y$ ein innerer Punkt einer dieser Kanten sein.
  \end{proof}
  
  Dieser Beweis ist ebenfalls in \cite[Satz 4.1.2]{fritsch} zu finden, allerdings in einer topologischen Variante mittels Jordanbögen. 
  
  Eine besondere Klasse von Graphen wollen wir noch hervorheben.
  
  \begin{definition}{normaler Graph}
   Ein Graph heißt \textit{normal}, wenn er ein regulärer, vollständiger Streckengraph ist, bei dem jedes Dreieck Rand eines Gebiets ist.
  \end{definition}
\end{section}