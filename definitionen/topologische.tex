\begin{section}{Topologische Definitionen}
 Ohne genauer auf die topologischen Grundlagen einzugehen, wollen wir trotzdem die Äquivalenz zwischen beiden aktuellen Formulierungen zeigen. Dazu bedarf es allerdings eines Mindestmaßes an Begrifflichkeiten, die im folgenden erläutert werden.
 
 Wesentlich für die Frage, ob eine Landkarte färbbar ist, ist die Frage, was eine solche eigentlich ist. Der Vollständigkeit halber ist also hier nochmals die Definition zu nennen, die bereits aus der Einleitung bekannt ist, sowie einige zusätzliche Begriffe.
 
 \begin{definition}{Landkarte, Kante, Ecke, neutrale Punkte, Neutralitätsmenge, Land}
  \-\ 
  \begin{itemize}
    \item Eine \textit{Landkarte $\mathcal{L}$} ist eine endliche Menge von Kanten in der Ebene $\mathbb{R}^2$ derart, dass der Durchschnitt von je zwei verschiedenen Kanten in $\mathcal{L}$ entweder leer oder ein gemeinsamer Randpunkt dieser Kanten ist. 
    \item Als \textit{Kante} bezeichnen wir einen Jordanbogen in $\mathcal{L}$ und $k_\mathcal{L}$ ist die Anzahl der Kanten von $\mathcal{L}$.
    \item Ein Punkt in $\mathbb{R}^2$ ist eine \textit{Ecke} $v_i$ von $\mathcal{L}$, wenn er Randpunkt eine Kante von $\mathcal{L}$ ist, und $v_\mathcal{L}$ ist die Anzahl der Ecken von $\mathcal{L}$.
    \item Ein \textit{neutraler Punkt} ist ein Punkt der zu einer Kante von $\mathcal{L}$ gehört.
    \item Die \textit{Neutralitätsmenge $N_{\mathcal{L}}$} von $\mathcal{L}$ ist die Menge aller ihrer neutralen Punkte. 
    \item Ein \textit{Land} von $\mathcal{L}$ ist eine Bogenkomponente des Komplements der Neutralitätsmenge von $\mathcal{L}$, das heißt, von $\mathbb{R}^2 \setminus N_{\mathcal{L}}$. Die Anzahl aller Länder ist mit $f_\mathcal{L}$ bezeichnet.
  \end{itemize}
 \end{definition}
 
 Das etwa die Neutralitätsmenge endlich ist oder was ein Jordanbogen ist, ist am besten der Literatur zu entnehmen (etwa \cite[Kapitel 2]{fritsch}). Um unser weiteres Vorgehen zu verstehen, sollte die Anschauung des Lesers allerdings ausreichen.
 
 Eine besondere Form von Landkarten wird für unsere späteren Betrachtungen besonders relevant sein:
 
 \begin{definitionl}{Reguläre Landkarte}{regland}
  Eine Landkarte $\mathcal{L}$ ist \textit{regulär}, wenn sie die folgenden Eigenschaften besitzt:
  \begin{itemize}
   \item Sie ist nicht leer,
   \item sie ist zusammenhängend,
   \item sie enthält keine Kante $B$, die zwei Komponenten verbindet, die in der Landkarte $\mathcal{L} \setminus B$ nicht verbunden wären,
   \item es gibt keinen Knoten, der Endpunkt von nur einer Kante ist,
   \item je zwei Länder haben höchstens eine gemeinsame Grenzlinie.
  \end{itemize}
 \end{definitionl}
 
 Zusätzlich brauchen wir noch ein weiteres Konstrukt, das sich direkt aus Landkarten ableiten lässt.

 \begin{definitionl}{duale Landkarte}{dualeKarte}
  Eine Landkarte $\mathcal{L}^*$ heißt \textit{dual} zu der Landkarte $\mathcal{L}$, wenn gilt:
  \begin{enumerate}
   \item keine Ecke von $\mathcal{L}^*$ ist ein neutraler Punkt von $\mathcal{L}$,
   \item jedes Land von $\mathcal{L}$ enthält genau eine Ecke von $\mathcal{L}^*$,
   \item zwei Ecken von $\mathcal{L}^*$ sind genau dann durch eine Kante in $\mathcal{L}^*$ verbunden, wenn sie in benachbarten Ländern liegen,
   \item eine Kante von $\mathcal{L}^*$ enthält nur Punkte der beiden Länder von $\mathcal{L}$, denen ihre Ecken angehören, und genau einen inneren Punkt einer gemeinsamen Grenzlinie dieser Länder.
  \end{enumerate}
 \end{definitionl}
 
 Das zu jeder Landkarte mit mindestens zwei Ländern stets eine duale Landkarte existiert, lässt sich ebenfalls \cite{fritsch} entnehmen. Bei der dualen Landkarte handelt es sich um ein Konstrukt, das von einem Graphen im kombinatorischen Sinne nicht mehr weit entfernt ist, wie nach der Lektüre der nächsten Abschnitte ersichtlich wird.
 
 Eine Besonderheit stellen folgende Landkarten dar.
 
 \begin{definitionl}{Kubische Landkarte}{kubisch}
  Eine Landkarte heißt \textit{kubisch}, wenn sie regulär ist und alle Ecken zwischen genau drei Ländern liegen.
 \end{definitionl}

 Für nichtleere und zusammenhängende Landkarten gilt die wohl bekannte \textit{Eulersche Polyederformel}
 \[v_\mathcal{L} - k_\mathcal{L} + f_\mathcal{L} = 2\]
 welche wir hier ohne Beweis angeben. Damit lässt sich die folgende Ungleichung zeigen, welche für den Beweis des Vier-Farben-Satzes von elementarer Bedeutung ist.
 
 \begin{satzl}{Summe der Knotengrade}{4.12}
  Für reguläre Landkarten gilt: \[\sum_{r=1}^{v_\mathcal{L}} (6-d_\mathcal{L}(v_r)) \geq 12\]
 \end{satzl}
 \begin{proof}
  Mit der Eulerschen Polyederformel gilt:
  \begin{align*}
    \sum_{r=1}^{v_\mathcal{L}} (6-d_\mathcal{L}(v_r)) &= 6\cdot v_\mathcal{L}-2\cdot k_\mathcal{L}\\
    &= 6\cdot v_\mathcal{L} - 6\cdot k_\mathcal{L} + 4\cdot k_\mathcal{L}\\
    &\geq 6\cdot v_\mathcal{L} - 6\cdot k_\mathcal{L} + 6\cdot f_\mathcal{L}\\
    &= 6\cdot (v_\mathcal{L} - k_\mathcal{L} + f_\mathcal{L})\\
    &= 12
  \end{align*}
 \end{proof}

 Diese Resultate lassen sich ebenfalls leicht auf die Strukturen der Kombinatorik anwenden.

\end{section}