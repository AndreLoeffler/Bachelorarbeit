\begin{section}{Die Birkhoff-Zahl}
 Die Birkhoff-Zahl wurde 1972 von Thomas Saaty in \cite{Saaty} eingeführt.
 
 \begin{definition}{Birkhoff-Zahl}
  Die \textit{Birkhoff-Zahl} ist die Anzahl der Länder, die eine Landkarte mindestens haben muss, um nicht mit vier Farben färbbar zu sein.
 \end{definition}
 
 Heute müsste die Birkhoff-Zahl also wohl auf $\infty$ gesetzt werden, da Guthries Vermutung als bewiesen angesehen werden kann. Stattdessen versteht man darunter eine vom Kalenderdatum abhängige Größe. Die Birkhoff-Zahl am Tag $t$ war $b$, wenn zum Zeitpunkt $t$ bewiesen war, dass ein Gegenbeispiel mindestens $b$ Länder haben muss. So lautet für die Jahre 1852 -- 1879 die Birkhoff-Zahl 6.
 
 Die Birkhoff-Zahl ist für das eigentliche Problem nur von geringer Bedeutung, da parallel dazu auch die Reduzibilitätstheorie verfeinert wurde. Sie wird hier nur genannt, um den historischen Verlauf des Problems darzustellen.
 
 \begin{figure}[hb]
  \label{birkhoffzahl}
  \centering
  \begin{tabular}{l | c | r}
   & t & b \\ \hline
   B.G. Weiske & vor 1852 & 6 \\
   A.B. Kempe & 1879 & 13 \\
   P. Franklin & 1922 & 26 \\
   C.N. Reynolds & 1926 & 28 \\
   P. Franklin & 1938 & 32 \\
   C.E. Winn & 1940 & 36 \\
   O. Ore \& J.G. Stemple & 1968 & 41 \\
   W.R. Stromquist & Juli 1973 & 45 \\
   J. Mayer & September 1973 & 48 \\
   W.R. Stromquist & 1974 & 52 \\
   J. Mayer & 1975 & 96 \\
  \end{tabular}
  \caption[Die Birkhoff-Zahl im Laufe der Zeit]{Die Birkhoff-Zahl im Laufe der Zeit}
 \end{figure}

 
\end{section}