\begin{section}{Topologisch-kombinatorischer Zusammenhang}
 \begin{satz}[Graphentheoretische Formulierung]
  Jeder planare Graph ohne Schleifen ist 4-färbbar.
 \end{satz}
 
 Diese Variante wirft einige Fragen nach Begrifflichkeiten auf, welche jedoch bei genauerer Betrachtung leicht verständlich sind. Zunächst müssen wir uns jedoch davon überzeugen, dass diese Aussagen auch tatsächlich äquivalent sind.
 
 Im nächsten Abschnitt werden wir uns zunächst den allgemeinen Definitionen widmen, die nötig sind um diese Problematik graphentheoretisch angehen zu können. Danach werfen wir einen Blick auf den älteren Beweis von Appel und Haken, um den Ausgangspunkt für die Arbeit von Robertson, Sanders, Seymour und Thomas darzulegen. Im Anschluss werden wir diese Arbeit nachvollziehen, indem die wesentlichen Schritte, die ``Reduktion'' und die ``Zwangsläufigkeit'', genauer beleuchtet werden. Abschließend werden noch einige Umformulierungen und Anwendungen des Vier-Farben-Satzes diskutiert.
\end{section}
