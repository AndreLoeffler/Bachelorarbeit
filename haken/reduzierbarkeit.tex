\begin{section}{Reduzierbarkeit}
 \label{sec:AHRed}
 Der wichtigste Unterschied zwischen den beiden Beweisen besteht in der Wahl der angewendeten Reduzierbarkeitstheorien. 
 
 Um die folgenden Theorien verstehen zu können, geben wir in aller Kürze noch weitere Definitionen.
 
 \begin{definition}{Reduzent}
  Sei $K$ eine Konfiguration mit den Außenecken $v_1,\cdots,v_r$ in zyklischer Anordnung. Ein Paar $(S,\sigma)$, bestehend aus einem Graphen $S$ und einer surjektiven Abbildung $\sigma$ von der Menge der Außenecken von $K$ auf die Außenecken von $S$ heißt \textit{Reduzent} von $K$, wenn $S$ weniger Ecken hat als $C$ und zusätzlich folgendes gilt:
  \begin{itemize}
   \item $\sigma$ erhält die Nachbarschaftsrelation, d.h. für alle $i = 1,\cdots,k$ sind $\sigma(v_i)$ und $\sigma(v_{i+1})$ ebenfalls benachbart.
   \item Urbilder verschiedener Außenecken von $S$ bezüglich $\sigma$ trennen sich nicht gegenseitig, d.h. für $i,j,k,l \in \{1,\cdots,k\}$ mit
   \[ \sigma(v_i) = \sigma(v_j) \neq \sigma(v_k) = \sigma(v_l)\]
   ist die Annordnung der Indizes $i < k < j < l$ nicht möglich.
  \end{itemize}
 \end{definition}
 
 \begin{definition}{Menge aller Randfärbungen, $\sigma$-richtig, $\sigma$-verträglich}
  Sei $K$ eine Konfiguration mit den Außenecken $v_1,\cdots,v_r$, $G$ ein minimales Gegenbeispiel $(S,\sigma)$ ein Reduzent von $K$ und $\Psi(S)$ die Menge aller zulässigen Eckenfärbungen von $S$. Die \textit{Menge aller Randfärbungen} für einen Graphen mit $r$ Randknoten nennen wir $\Psi(r)$
  \begin{itemize}
   \item $K$ ist nur dann \textit{$\sigma$-richtig} in $G$ eingebettet, wenn zwei Außenecken von $K$, die von $\sigma$ gleich abgebildet werden, in $G$ nicht benachbart sind. 
   \item Eine Färbung $\vartheta$ der Außenecken von $K$ heißt \textit{$\sigma$-verträglich}, wenn sie als Abbildung von der Form $\vartheta = \chi \circ \sigma$ für ein $\chi \in \Psi(S)$ ist.
   \item Die Menge der $\sigma$-verträglichen Randfärbung bezeichnen wir mit $\Phi(r,\rho)$.
   \item Mit $\Psi_0(K)$ bezeichnen wir die Menge aller von Anfang an guten Randfärbungen.
  \end{itemize}
 \end{definition}

 
 Diese Definition war vorher nicht nötig, da \rsst\-\ ohne Reduktionen auskommen, die einen Reduzenten benötigen. Es gibt im wesentlichen vier Wege, zu zeigen, dass in einem minimalen Gegenbeispiel keine Konfigurationen auftreten:
 
 \begin{definition}{A-,B- und C-Reduzibilität}
  \begin{itemize}
   \item \textbf{A-Reduzibilität}, benannt nach A. Errera\\
   Eine Konfiguration $K$ heißt \textit{A-reduzibel}, wenn sie einen Reduzenten besitzt, der die folgenden Eigenschaften besitzt:
   \begin{itemize}
    \item $K$ kann in eine Minimaltriangulation nur $\sigma$-richtig eingebettet werden und
    \item jede $\sigma$-verträgliche Randfärbung ist direkt durchfärbbar, das heißt, es gilt: $\Phi(r,\sigma) \subset \Phi_0(K)$
   \end{itemize}

   \item \textbf{B-Reduzibilität}, benannt nach Birkhoff\\
   \item \textbf{C-Reduzibilität}, benannt nach C.E. Winn\\
   \item \textbf{D-Reduzibilität}\\
   Von Heesch selbst entwickelt in \cite{heesch} veröffentlicht, nur um die zufällige alphabetische Reihenfolge einzuhalten D-Reduzibilität genannt. Sie bildet den Kern  des Reduktionsschritts von \rsst\-\ und wurde deshalb schon in Definition \ref{dred} erläutert. In diesem Zusammenhang ist allerdings noch der Dürre-Heesch-Algorithmus zu nennen, der eine D-Reduktion durchführt. Eine Beschreibung des Algorithmus mit Teilen des Quellcodes findet sich etwa in \cite[Kapitel 6.4]{fritsch}.
  \end{itemize}
 \end{definition}

\end{section}
