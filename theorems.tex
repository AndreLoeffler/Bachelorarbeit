\newtheoremstyle{style}
   {0cm}                 %Space above
   {0cm}                   %Space below
   {\normalfont}           %Body font: original {\normalfont}
   {}                      %Indent amount \left(empty = no indent,
                           %\parindent = para indent\right)
   {\normalfont\bfseries}  %Thm head font original {\normalfont\bfseries}
   {:}                     %Punctuation after thm head original :
   {\newline}              %Space after thm head: " " = normal interword
                           %space; \newline = linebreak
  {\textbf{\thmname{#1}\thmnumber{ #2}}\thmnote{ (#3)}}
                           %Hier wird die endgültige Struktur deiner Umgebung festgelegt. Hier funktioniert vspace.                    
                           %Thm head spec \left(can be left empty, meaning
                           %`normal'\right) original {\underline{\thmname{#1}\thmnumber{ #2}\thmnote{ \left(#3\right)}}}

\theoremstyle{style}

\newtheorem{definitioncont}{Definition}
\newtheorem{satzcont}{Satz}[chapter] % restart counter at 1 for each chapter
\renewcommand{\thesatzcont}{\arabic{chapter}.\arabic{satzcont}} % only show chapter.satz
\newtheorem{lemmacont}[satzcont]{Lemma} % Lemma numbering together with theorem
\newtheorem{corollarycont}[satzcont]{Korollar} % Corollary numbering together with theorem
\newtheorem{propositioncont}[satzcont]{Proposition} % Proposition numbering together with theorem

%define new environments for indentations to the right
%with labels
\newenvironment{satzl}[2]{\vspace{-20pt}\begin{adjustwidth}{20pt}{}\begin{satzcont}[#1]\label{#2}}{\end{satzcont}\end{adjustwidth}}
\newenvironment{lemmal}[2]{\vspace{-20pt}\begin{adjustwidth}{20pt}{}\begin{lemmacont}[#1]\label{#2}}{\end{lemmacont}\end{adjustwidth}}
\newenvironment{corollarl}[2]{\vspace{-20pt}\begin{adjustwidth}{20pt}{}\begin{corollarycont}[#1]\label{#2}}{\end{corollarycont}\end{adjustwidth}}
\newenvironment{propositionl}[2]{\vspace{-20pt}\begin{adjustwidth}{20pt}{}\begin{propositioncont}[#1]\label{#2}}{\end{propositioncont}\end{adjustwidth}}
\newenvironment{definitionl}[2]{\vspace{-20pt}\begin{adjustwidth}{20pt}{}\begin{definitioncont}[#1]\label{#2}}{\end{definitioncont}\end{adjustwidth}}
%without labels
\newenvironment{satz}[1]{\vspace{-20pt}\begin{adjustwidth}{20pt}{}\begin{satzcont}[#1]}{\end{satzcont}\end{adjustwidth}}
\newenvironment{lemma}[1]{\vspace{-20pt}\begin{adjustwidth}{20pt}{}\begin{lemmacont}[#1]}{\end{lemmacont}\end{adjustwidth}}
\newenvironment{corollar}[1]{\vspace{-20pt}\begin{adjustwidth}{20pt}{}\begin{corollarycont}[#1]}{\end{corollarycont}\end{adjustwidth}}
\newenvironment{proposition}[1]{\vspace{-20pt}\begin{adjustwidth}{20pt}{}\begin{propositioncont}[#1]}{\end{propositioncont}\end{adjustwidth}}
\newenvironment{definition}[1]{\vspace{-20pt}\begin{adjustwidth}{20pt}{}\begin{definitioncont}[#1]}{\end{definitioncont}\end{adjustwidth}}


%define node commands for tikz path environment
\newcommand{\blacknode}{node[circle,fill=black,inner sep=2pt]}                                                 %gamma 5
\newcommand{\smallnode}{node[circle,fill=black,inner sep=0.1pt]}                                         %gamma 6
\newcommand{\whitenode}{node[circle,inner sep=2pt,draw]}                                                 %gamma 7
\newcommand{\rectanglenode}{node[shape=rectangle,draw,inner sep=2.5pt]}                                 %gamma 8
\newcommand{\threenode}{node[regular polygon,regular polygon sides=3,inner sep=1.5pt,rotate=180]}        %gamma 9
\newcommand{\fivenode}{node[regular polygon,regular polygon sides=5,inner sep=2pt]}                        %gamma 10

\newcommand{\rsst}{Robertson, Sanders, Seymour und Thomas}
\newcommand{\itP}{\mathscr{P}}
\newcommand{\itC}{\mathscr{C}}
\newcommand{\itA}{\mathscr{A}}
\newcommand{\itD}{\mathscr{D}}
\newcommand{\itE}{\mathscr{E}}