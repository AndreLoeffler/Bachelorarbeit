
\newtheoremstyle{style}
   {0.5cm}                 %Space above
   {0cm}                   %Space below
   {\normalfont}                      %Body font: original {\normalfont}
   {}                      %Indent amount \left(empty = no indent,
                           %\parindent = para indent\right)
   {\normalfont\bfseries}  %Thm head font original {\normalfont\bfseries}
   {:}                     %Punctuation after thm head original :
   {\newline}              %Space after thm head: " " = normal interword
			   %space; \newline = linebreak
  {\textbf{\thmname{#1}\thmnumber{ #2}}\thmnote{ (#3)}}
%Hier wird die endgültige Struktur deiner Umgebung festgelegt. Hier funktioniert vspace.                    
                                        %Thm head spec \left(can be left empty, meaning
                           %`normal'\right) original {\underline{\thmname{#1}\thmnumber{ #2}\thmnote{ \left(#3\right)}}}

\theoremstyle{style}

\newtheorem{definition}{Definition}
\newtheorem{satz}{Satz}
\newtheorem{lemma}[satz]{Lemma} % Lemma numbering together with theorem
\newtheorem{corollary}[satz]{Korollar} % Corollary numbering together with theorem
\newtheorem{proposition}[satz]{Proposition} % Proposition numbering together with theorem