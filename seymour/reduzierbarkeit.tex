\begin{section}{Reduzierbarkeit}
 \begin{definition}{freie Vervollständigung}
  Sei $K$ eine Konfiguration. Eine Beinahe-Triangulation $S$ heißt \textit{freie Vervollständigung} von $K$ \textit{mit dem Ring $R$}, wenn
  \begin{enumerate}[i)]
   \item $R$ ein induzierter Ring von $S$ ist, der die Außenfacette von $S$ begrenzt,
   \item $G(K)$ ein induzierter Teilgraph von $S$ ist, $G(K) = S \setminus V(R)$ gilt, jede Facette von $G(K)$ auch eine Facette von $S$ ist, die Außenfacette von $G(K)$ den Ring $R$ und die Außenfacette von $S$ beinhaltet,
   \item jeder Knoten $v$ von $S$, der nicht in $V(R)$ liegt, in $S$ Knotengrad $\gamma_K(v)$ hat.
  \end{enumerate}
 \end{definition}
 
 %TODO: formulieren!%
 Man kann leicht überprüfen, dass jede Konfiguration eine freie Vervollständigung hat. (Hier benutzen wir den Umstand, dass in der Definition von Konfiguration die Ringgröße $\geq 2$ ist -- die Ringgröße ist genau die Länge des Rings in der Freien Vervollständigung, wie der Leser nachprüfen kann.) Gibt es weiterhin zwei freie Vervollständigungen $S_1, S_2$ von $K$, so existiert ein Homeomorphismus, der $G(K)$ punktweise fixiert und $S_1$ auf $S_2$ abbildet. Dazu verwendet man Eigenschaft i) aus der Definition von Konfiguration. Also gibt es eigentlich nur eine freie Vervollständigung, weswegen wir ohne Unklarheiten von \textit{der} freien Konfiguration sprechen können.
 
 Sei $R$ ein Kreis. Es gibt das Konzept der \textit{Kontinuität} für eine Menge von 4-Färbungen von $R$, welches auf Kempe \cite{AmJMath79} und Birkhoff \cite{AmJMath35} zurückgeht. Wir benötigen hier nicht das vollständige Konzept, sondern nennen nur die Eigenschaften, die wir brauchen. Sie lauten:

\end{section}
