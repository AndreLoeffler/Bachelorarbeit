\begin{section}{Reduzierbarkeit}
 Wir wollen konsistente Kantenfärbungen definieren. Dazu beginnen diesen Abschnitt mit den dazu nötigen, hinführenden Definitionen:
 \begin{definition}{Kantenfärbung, Match, signiertes Match, signiertes Matching, $\theta$-Passend}
  Sei $R$ ein Kreis und $\theta \in \{-1,0,1\}$.
  \begin{itemize}
   \item Eine \textit{Kantenfärbung} von $R$ ist eine Abbildung $\kappa: E(R) \mapsto \{-1,0,1\}$.
   \item Ein \textit{Match} $m$ ist eine Menge von verschiedenen Kanten $\{e,f\}$ aus $R$. 
   \item Ein \textit{signiertes Match} (engl.: signed match) $(m,\mu)$ ist ein Paar aus einem Match $m$ und $\mu = \pm 1$.
   \item Ein \textit{signiertes Matching} ist eine Menge $M$ von signierten Matches, sodass für unterschiedliche $(\{e,f\},\mu),(\{e',f'\},\mu') \in M$ gilt:
   \begin{enumerate}[(i)]
    \item $\{e,f\}\cap\{e',f'\} = \emptyset$ und
    \item nach dem Löschen von $e'$ und $f'$ liegen $e$ und $f$ in der gleichen Zusammenhangskomponente von $R$.
   \end{enumerate}
   Ist $M$ ein signiertes Matching, so ist $E(M) := \{e\in E(R) | e\in m $ für ein $(m,\mu) \in M\}$.
   \item Eine Kantenfärbung $\kappa$ von $R$ heißt \textit{$\theta$-passend} für ein signiertes Matching $M$ in $R$, wenn gilt:
   \begin{enumerate}[(i)]
    \item $E(M) = \{e \in E(R) | \kappa(e) \neq \theta\}$ und
    \item für alle $(\{e,f\},\mu) \in M$ gilt: $\kappa(e) = \kappa(f) \Leftrightarrow \mu = 1$.
   \end{enumerate}
  \end{itemize}
 \end{definition}

 Nun können wir die eigentlich gesuchte Definition aufstellen.
 
 \begin{definition}{konsistente Kantenfärbung}
  Sei $K$ ein Kreis und $\theta \in \{-1,0,1\}$. Eine Menge $\mathcal{C}$ von Kantenfärbungen von $K$ heißt \textit{konsistent}, wenn für jedes $\kappa \in \mathcal{C}$ und jedes mögliche $\theta$ ein signiertes Matching $M$ existiert, so dass $\kappa$ $\theta$-passend für $M$ ist, und $\mathcal{C}$ jede Kantenfärbung, die für $M$ $\theta$-passend ist, enthält.
 \end{definition}

 Für das nächste Teilresultat benötigen wir noch diese Definitionen.
 
 \begin{definition}{Verpackung, Aufzug}
  Sei $H$ eine Beinahe-Triangulation. 
  \begin{itemize}
   \item Dann gibt es einen geschlossenen Pfad $(v_0,v_1,\cdots,v_k)$ durch die zur Außenfacette inzidenten Knoten. Dann existiert ein Kreis $R$ der Länge $k$ mit Kanten $e_1,\cdots e_k$, nicht notwendigerweise ein Kreis in $H$. Für $i \leq i \leq k$ definieren wir einen Zeiger $\phi(e_i) := f_i$, wobei $f_i$ die Kante zwischen $v_{i-1}$ und $v_i$ aus $H$ ist. Wir sagen dann, $\phi$ \textit{verpackt $H$ in $R$}. 
   \item Ist $\kappa$ eine Trifärbung von $H$, so setzen wir für alle $e \in E(R): \lambda(e) = \kappa(\phi(e))$. Dann ist $\lambda$ eine Kantenfärbung von $R$ und wir nennen $\lambda$ einen \textit{Aufzug} von $\kappa$ (durch $\phi$).
  \end{itemize}
 \end{definition}
 
 \begin{satzl}{Konsistente Aufzüge}{constLift}
  Sei $H$ eine Beinahe-Triangulation und $R$ ein Kreis, in den $H$ durch $\phi$ verpackt ist. Sei $\mathcal{C}$ die Menge aller Aufzüge von $\phi$ von Trifärbungen von $H$. Dann ist $\mathcal{C}$ konsistent.
 \end{satzl}

\end{section}