\begin{section}{Reduzierbarkeit}
 \begin{definition}[Freie Vervollständigung]
  Sei $K$ eine Konfiguration. Eine Beinahe-Triangulation $S$ heißt \textit{Freie Vervollständigung} von $K$ \textit{mit dem Ring $R$}, wenn
  \begin{enumerate}[i)]
   \item $R$ ein induzierter Ring von $S$ ist, der die Außenfacette von $S$ begrenzt,
   \item $G(K)$ ein induzierter Teilgraph von $S$ ist, $G(K) = S \setminus V(R)$ gilt, jede Facette von $G(K)$ auch eine Facette von $S$ ist, die Außenfacette von $G(K)$ den Ring $R$ und die Außenfacette von $S$ beinhaltet,
   \item jeder Knoten $v$ von $S$, der nicht in $V(R)$ liegt, in $S$ Knotengrad $\gamma_K(v)$ hat.
  \end{enumerate}
 \end{definition}
 
 Man kann leicht überprüfen, dass jede Konfiguration eine Freie Vervollständigung hat. (Hier benutzen wir den Umstand, dass in der Definition von Konfiguration die Ringgröße $\geq 2$ ist -- die Ringgröße ist genau die Länge des Rings in der Freien Vervollständigung, wie der Leser nachprüfen kann.)

\end{section}
