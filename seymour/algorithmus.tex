\begin{section}{Der Algorithmus}
 In diesem Abschnitt gehen wir soweit auf den von \rsst\-\  angegebenen Algorithmus ein, wie wir ihn für den Beweis von Satz \ref{2.1} benötigen.
 
 Dazu gehen wir weiter auf die bereits von Birkhoff gezeigten Resultate aus \cite{AmJMath35} ein. Zunächst brauchen wir jedoch noch eine Definition.
 
 \begin{definition}{Äquivalente Kantenfärbungen}
  Sei $R$ ein Kreis und $\kappa,\kappa'$ zwei Kantenfärbungen von $R$. $\kappa$ und $\kappa'$ heißen \textit{äquivalent}, wenn eine Permutation $\lambda$ von $\{-1,0,1\}$ existiert, sodass für jede Kante $e\in E(R)$ gilt: $\kappa'(e) = \lambda(\kappa(e))$.
 \end{definition}

 Zur Vereinfachung führen wir eine neue Schreibweise für Kantenfärbungen ein. Sei dazu $R$ ein Kreis der Länge $n$, die Zahlen $k_i \in \{-1,0,1\}$ für alle $i = 1,\cdots,n$ und sei $\kappa = (k_1,k_2,\cdots,k_n)$, derart dass $\kappa(e_i) = k_i$ für jede Kante $e_i$ des Kreises $R$.
 
 \begin{lemmal}{ }{6.3}
  Sei $R$ ein Kreis der Länge 4 mit den Kanten $e_1,e_2,e_3,e_4$, in dieser Reihenfolge. Sei $\itC_0$ die Menge aller Kantenfärbungen, die zu $(0,0,0,0)$ äquivalent sind. In gleicher Weiße seien $\itC_1,\itC_2,\itC_3$ die Mengen der Kantenfärbungen, die zu $(0,1,1,0),(0,1,0,1),(0,0,1,1)$ äquivalent sind. Dann beinhaltet jede nichtleere konsistente Menge von Kantenfärbungen von $R$ eine der Mengen $\itC_0 \cup \itC_1$, $\itC_1 \cup \itC_2$, $\itC_2 \cup \itC_3$ oder $\itC_3 \cup \itC_0$.
 \end{lemmal}
 
 Der Beweis für dieses Lemma bleibt dem Leser überlassen, da er nicht sehr anspruchsvoll ist.
 
 \begin{lemmal}{ }{6.4}
  Sei $R$ ein Kreis der Länge 5 mit den Kanten $e_1,e_2,e_3,e_4,e_5$ in dieser Reihenfolge. Für $1 \leq i \neq j \leq 5$ sei $\itA_{ij}$ definiert als die Äquivalenzklasse der Kantenfärbungen von $R$, äquivalent zu einer Kantenfärbung $\kappa$ mit 
  \[\kappa(e_l) = \begin{cases}
    1 &\text{falls } l = i\\
    -1 &\text{falls } l = j\\
    0 &\text{sonst.}
   \end{cases}\]
  Für $1 \leq i \leq 5$ sei $\itC_i = \itA_{ij}\cup\itA_{ik}\cup\itA_{jk}$, wobei die Kanten $e_j,e_k$, mit $e_j,e_k \neq e_i$, die beiden Kanten sind, die je genau einen Endpunkt mit $e_i$ gemeinsam haben. Seien weiter für $1\leq i \leq 5$ die Kanten aus $R$, die zu $e_i$ unterschiedlich sind, der Reihe nach mit $e_a,e_b,e_c,e_d$ bezeichnet und sei  $\itD_i = \itA_{ac}\cup\itA_{ad}\cup\itA_{bc}\cup\itA_{bd}$. Sei weiter $\itE = \itA_{12}\cup\itA_{23}\cup\itA_{34}\cup\itA_{45}\cup\itA_{51}$.\\
  Jede nichtleere konsistente Menge, die $\itE$ genügt, beinhaltet eine der Mengen 
  \[\itC_1,\cdots,\itC_5.\itD_1,\cdots,\itD_5,\itE\text{.}\]
 \end{lemmal}
 \begin{proof}
  Sei $\itC$ eine konsistente Menge.
  \begin{enumerate}[(i)]
   \item Wenn $\itA_{12} \subseteq \itC$ gilt, so beinhaltet $\itC$ eine der Mengen $\itA_{13},\itA_{15}$ und eine der Mengen $\itA_{23},\itA_{25}$.\\
   Sei nun $\kappa = (-1,1,0,0,0) \in \itC$. Da $\itC$ konsistent ist, existiert ein signiertes Matching $M$ derart, dass $\kappa$ $(-1)$-passend für $M$ ist und $\itC$ beinhaltet alle Kantenfärbungen, die $(-1)$-passend für $M$ sind. Da $\kappa$ $(-1)$-passend für $M$ ist, ist $M$ also eines der beiden:
   \[ \{(\{e_2,e_3\},-1),(\{e_4,e_5\},1)\} \text{ oder } \{(\{e_2,e_5\},-1),(\{e_3,e_4\},1)\} \]
   Ist $M$ das erstere, so ist die Kantenfärbung $(-1,0,1,0,0)$ für $M$ $(-1)$-passend und somit ein Element aus $\itC$. Daher gilt dann auch $\itA_{13}\subseteq\itC$. An sonsten ist $(-1,0,0,0,1)$ für $M$ $(-1)$-passend, also $\itA_{15}\subseteq\itC$. In ähnlicher Weise gibt es nur zwei signierte Matchings $M'$, für die $\kappa$ $1$-passend ist. Daraus lässt sich analog ableiten, dass entweder $\itA_{23}\subseteq\itC$ oder $\itA_{25}\subseteq\itC$.
   \item Wenn $\itA_{13} \subseteq \itC$ gilt, so beinhaltet $\itC$ eine der Mengen $\itA_{23},\itA_{35}$.\\
   Dieser Teil läuft ebenfalls sehr ähnlich zu den bereits geführten Schritten.
  \end{enumerate}
  Sei nun $\itC$ eine konsistente Menge, die $\itE$ genügt. Wir können annehmen, dass eine der Mengen $\itA_{12},\itA_{23},\itA_{34},\itA_{45},\itA_{15}$. Weiter gehen wir davon aus, dass nicht alle dieser Mengen in $\itC$ sind, da sonst $\itE\subseteq\itC$ gelten würde, wie gefordert. Aus Gründen der Symmetrie nehmen wir an, dass $\itA_{12}\subseteq\itC$ und $\itA_{23}\not\subseteq\itC$. Aus (i) können wir nun annehmen, dass $\itA_{25}\subseteq\itC$ gilt. Wenn auch $\itA_{15}\subseteq\itC$ gilt, so ist $\itC_1\subseteq\itC$, wie gefordert. Also nehmen wir weiter an, dass $\itA_{15}\not\subseteq\itC$. Ebenfalls nach (i) ist dann $\itA_{13}\subseteq\itC$. Wegen (ii) gilt dann $\itA_{35}\subseteq\itC$ und somit auch $\itD_4\subseteq\itC$.
 \end{proof}
 
 Der von \rsst\-\ angegebene Algorithmus durchsucht die zu färbende Triangulation $T$ nach Kurzkreisen. Tritt ein Kurzkreis in $T$ auf, so verwendet er den Algorithmus, dem \ref{3.1} zu Grunde liegt, teilt $T$ anhand des Kreises in einen inneren und einen äußeren Teil und bestimmt für den inneren alle möglichen Färbungen. Dass unter diesen Färbungen tatsächlich eine ist, die zusammen mit der Färbung des äußeren Teils eine gültige Färbung von ganz $T$ ergibt, zeigen die obigen Resultate.
 
 Damit darf also eine Triangulation, die ein Gegenbeispiel sein soll, keine Kurzkreise enthalten und ist somit intern 6-fach zusammenhängend.
\end{section}