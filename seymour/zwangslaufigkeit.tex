\begin{section}{Zwangsläufigkeit}
 Dieser Abschnitt widmet sich dem Beweis von Satz \ref{2.3}.
 
 \begin{definition}{Wagenrad, Radnabe}
  Eine Konfiguration $W$ heißt \textit{Wagenrad} (engl.: cartwheel), wenn es einen Knoten $w$, genannt \textit{Radnabe} (engl.: hub), und zwei Kreise $C_1, C_2$ mit folgenden Eigenschaften gibt:
  \begin{enumerate}[(i)]
   \item es gilt $\{w\} \cap C_1 = \emptyset,\{w\} \cap C_2 = \emptyset$ und $C_2 \cap C_1 = \emptyset$, aber $\{w\} \cup C_1 \cup C_2 = V(G(W))$,
   \item $C_1$ und $C_2$ sind Teilgraphen von $G(W)$ und die Außenfacette von $C_2$ ist die Außenfacette von $G(W)$,
   \item $w$ ist adjazent zu allen Knoten von $C_1$ aber keinem Knoten von $C_2$.
  \end{enumerate}
 \end{definition}

 Es gibt folglich also vier Sorten von Kanten in einem Wagenrad: Kanten von $C_1$, Kanten von $C_2$, Kanten zwischen $w$ und und $C_1$ sowie Kanten zwischen $C_1$ und $C_2$. Zur Veranschaulichung noch eine Darstellung eines Wagenrades.
 
 \[ \begin{tikzpicture}
 \draw
    (0,1.5) \smallnode(a1){} (0,2) \rectanglenode(a2){} (0,4) \blacknode(a3){}
    (0.5,1) \smallnode(b1){}
    (1,0.5) \smallnode(c1){} (1,3) \blacknode(c2){}
    (2,0) \smallnode(d1){} (2,1.5) \rectanglenode(d2){} (2,5) \blacknode(d3){} (2,6) \rectanglenode(d4){} 
    (3,1) \blacknode(e1){} (3,3) \rectanglenode(e2){w} (3,5.5) \blacknode(e3){} 
    (4,0) \smallnode(f1){} (4,1.5) \smallnode(f2){} (4,5) \blacknode(f3){} (4,6) \blacknode(f4){} 
    (5,3) \whitenode(g1){} (5,5) \rectanglenode(g2){}
    (5.5,1) \whitenode(h1){} 
    (6,2) \smallnode(i1){} (6,3) \smallnode(i2){} (6,4) \smallnode(i3){};
    \filldraw (a1) -- (b1) -- (c1) -- (d1) -- (f1) -- (h1) -- (i1) -- (i2) -- (i3) -- (g2) -- (f4) -- (d4) -- (a3) -- (a2) -- (a1);
    \filldraw (c2) -- (d2) -- (e1) -- (f2) -- (g1) -- (f3) -- (e3) -- (d3) -- (c2) -- (e2) -- (g1);
    \filldraw (e1) -- (e2) -- (e3);
    \filldraw (f3) -- (e2) -- (d2) -- (d1) -- (e1) -- (f1) -- (f2) -- (e2) -- (d3);
    \filldraw (b1) -- (d2) -- (c1);
    \filldraw (a1) -- (d2) -- (a2) -- (c2) -- (a3) -- (d3) -- (d4) -- (e3) -- (f4) -- (f3) -- (g2) -- (g1) -- (i3);
    \filldraw (i2) -- (g1) -- (i1) -- (f2) -- (h1);
\end{tikzpicture}     \]
\end{section}
