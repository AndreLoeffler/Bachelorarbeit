\begin{section}{Konfigurationen}
 Ein minimales Gegenbeispiel ist ein planarer schleifenfreier Graph $G$, der nicht 4-färbbar ist, derart dass aber jeder planaren Graphen $G'$ mit $|V(G')| + |E(G')| < |V(G)| + |E(G)|$ eine gültige 4-Färbung besitzt. Unser Ziel ist also, zu zeigen, dass es keinen solchen Graphen $G$ geben kann. 
   
 Aus den allgemeinen Vorüberlegungen wissen wir, dass jedes minimale Gegenbeispiel nur Knoten mit Grad mindestens fünf hat. So sieht man leicht, dass jedes minimale Gegenbeispiel 5-fach knotenzusammenhängend und bis auf die Knoten mit Grad genau fünf sogar 6-fach. Wir nennen einen Kreis $C$ in einer Triangulation mit fünf oder weniger Knoten einen \textit{Kurzkreis} (engl.: short circuit), wenn sowohl Knoten im inneren als auch im äußeren Bereich von $C$ liegenund wenn für genau fünf Knoten in jedem Bereich mindestens zwei Knoten liegen. Wir nennen eine Triangulation intern 6-fach zusammenhängend, wenn sie keine Kurzkreise beinhaltet. Das führt zu folgendem Resultat:
  
 \begin{satzl}{}{4.1}
  Jedes minimale Gegenbeispiel ist eine intern 6-fach zusammenhängende Triangulation. 
 \end{satzl}

 Um genauer zu verstehen, warum der Graph 6-fach zusammenhängend sein muss, empfiehlt sich die Lektüre von \cite{AmJMath35}. Birkhoff schaffte es bereits 1913 zu zeigen, dass schwächer zusammenhängende Konfigurationen reduzierbar und damit vierfärbbar sind. Dazu bediente er sich der Resultate von A. B. Kempe -- Ketten, die mit einer beschränkten Auswahl an Farben färbbar sind, und Ringen, die eine Karte in eine innere und eine äußere Region teilen. 
 
 Später werden wir einen Algorithmus kennenlernen, der eine Färbung für eine Triangulation mit Kurzkreis liefert. Daraus lässt sich dann auch der Beweis für Satz \ref{4.1} ableiten.
 
 Als nächstes folgt eine Definition, die elementar für unseren Beweis ist. Der Begriff der Konfiguration taucht ebenfalls bei Appel \& Haken auf, jedoch besitzen sie dort andere Eigenschaften.
  
 \begin{definition}{Konfiguration}
  Eine Konfiguration $K$ besteht aus einer Beinahe-Triangulation $G$ und einer Zuordnung $\gamma_K : V(G) \mapsto \mathbb{Z}_+$ mit folgenden Eigenschaften:
  \begin{enumerate}[i)]
   \item Für jeden Knoten $v$ besteht $G(K) \setminus \{v\}$ höchstens zwei Zusammenhangskomponenten. Gibt es genau zwei, so ist $\gamma_K(v) = d_G(v) + 2$.
   \item Für jeden Knoten $v$, der nicht zur Außenfacette inzident ist, gilt $\gamma_K(v) = d_G(v)$. Für die anderen Knoten $v'$ gilt $\gamma_K(v') > d_G(v')$. In beiden Fällen gilt zusätzlich $\gamma_K(v) \geq 5$.
   \item $K$ hat Ringgröße $\geq 2$. Die \textit{Ringgröße} von $K$ ist definiert als $\sum_v (\gamma_K(v) - d_G(v) - 1)$ für alle Knoten $v$, die zur Außenfacette inzident sind und für die $G(K) \setminus \{v\}$ zusammenhängend ist.
  \end{enumerate}
 \end{definition}
 
 Um $\gamma_K$ für jeden Knoten in einer planaren Zeichnung von $G$ darzustellen, gibt es mehrere Möglichkeiten. Die Offensichtliche wäre natürlich, neben jedem Knoten seinen Wert zu notieren, was jedoch sehr schnell unübersichtlich wird. Stattdessen werden wir unseren Knoten verschiedene Formen geben, wie der folgenden Übersicht zu entnehmen ist.
 
 \[ \begin{tikzpicture}
    \tikzstyle{ann} = [draw=none,fill=none,right]
    \matrix[nodes={draw},column sep=0.5cm] {
    \node[circle,fill=black,inner sep=2pt] {}; &
    \node[draw=none,fill=none] {$\gamma_K(v) = 5$}; \\
    \node[circle,fill=black,inner sep=0.1pt] {}; &
    \node[draw=none,fill=none] {$\gamma_K(v) = 6$}; \\
    \node[circle,inner sep=2pt] {}; &
    \node[draw=none,fill=none] {$\gamma_K(v) = 7$}; \\
    \node[rectangle,inner sep=2.5pt] {}; &
    \node[draw=none,fill=none] {$\gamma_K(v) = 8$}; \\
    \node[regular polygon,regular polygon sides=3,inner sep=1.5pt] {};&
    \node[draw=none,fill=none] {$\gamma_K(v) = 9$}; \\
    \node[regular polygon,regular polygon sides=5,inner sep=2pt] {};&
    \node[draw=none,fill=none] {$\gamma_K(v) = 10$}; \\
    };
\end{tikzpicture}\]
 
 Zwei Konfigurationen $K$ und $L$ heißen \textit{isomorph}, falls ein Homeomorphismus existiert, der $G(K)$ auf $G(L)$ und $\gamma_L$ auf $\gamma_L$ abbildet. Später werden wir eine Menge aus 633 Konfigurationen betrachten, die für diesen Beweis essenziell sind. Jede Konfiguration, die zu einer dieser 633 isomorph ist, bezeichnen wir als \textit{gut}. 
   
 Um das eigentliche Problem zu beweisen, teilen wir die Suche nach einem minimalen Gegenbeispiel weiter auf. Somit ergeben sich diese beiden Aussagen:
   
 \begin{satzl}{}{4.2}
  Wenn $T$ ein minimales Gegenbeispiel ist, enthält $T$ keine gute Konfiguration.
 \end{satzl}
   
 \begin{satzl}{}{4.3}
  In jeder intern 6-fach zusammenhängenden Triangulation $T$ lässt sich eine gute Konfiguration finden.
 \end{satzl}
   
 Kombiniert man die Aussagen der Sätze \ref{4.1}, \ref{4.2} und \ref{4.3}, so sieht man, dass es kein minimales Gegenbeispiel geben kann und damit der Vier-Farben-Satz wahr ist. Auf \ref{4.2} werden wir im nächsten Abschnitt genauer eingehen, gefolgt von einem Abschnitt über \ref{4.3}. 
\end{section}