\begin{chapter}{Der Beweis von Appel und Haken}
  Nun widmen wir uns der Grundlage, auf die \rsst\-\ aufbauten, als sie begannen, ihren Beweis zu entwickeln. Die Arbeit von Appel \& Haken stammt aus dem Jahr 1976, zuerst von ihnen veröffentlicht im Illinois Journal of Mathematics -- \cite{AH1} und \cite{AH2}. Beschäftigte man sich mit dem Vier-Farben-Problem, so kannte man diese Arbeit. Leider weißt sie zwei charakteristische Schwächen auf: Teile des Beweises sind nicht von Hand überprüfbar, sondern nur von einem Computer, und selbst der Teil, der von Hand durchführbar ist, ist so lang und komplex aufgebaut, dass es einer einzelnen Person sehr schwer fallen würde, dies auch zu tun.
  
  Natürlich versuchten trotzdem Mathematiker weltweit, den Beweis von Appel \& Haken zu prüfen und so den Vier-Farben-Satz zu verifizieren -- so auch \rsst. In der Einleitung ihres Beweises schrieben sie:\\  
  "We began by trying to read the A\&H (Appel \& Haken) proof, but very soon gave this up. To check that the members of their 'unavoidable set' were all reducible would require a considerable amount of programming, and \textit{also} would require us to input by hand into the computer descriptions of some 1400 graphs; and this was not even part of their proof that was most controversial. We decided it would be easier, and more fun, to make up our own proof, using the same general approach as A\&H. So we did."\cite[Einleitung]{FourRSST}
  
  Zunächst betrachten wir nacheinander, die wesentlichen Bausteine des Beweises von Appel \& Haken. Dazu wollen wir nicht zu sehr ins Detail gehen, sondern nur so weit wie nötig ist, die Verbesserungen der Methoden durch \rsst\-\ zu erkennen. 
  
  \begin{section}{Reduzierbarkeit}
 \begin{definition}{freie Vervollständigung}
  Sei $K$ eine Konfiguration. Eine Beinahe-Triangulation $S$ heißt \textit{freie Vervollständigung} von $K$ \textit{mit dem Ring $R$}, wenn
 \end{definition}
 
 %TODO: formulieren!%
 
 
 Sei $R$ ein Kreis. Es gibt das Konzept der \textit{Kontinuität} für eine Menge von 4-Färbungen von $R$, welches auf Kempe \cite{AmJMath79} und Birkhoff \cite{AmJMath35} zurückgeht. Wir benötigen hier nicht das vollständige Konzept, sondern nennen nur die Eigenschaften, die wir brauchen. Sie lauten:

\end{section}

    \begin{section}{Die Konfigurationen}
   Eine Klasse von Graphen ist für den Beweis des Vier-Farben-Satzes wesentlich: Die Konfigurationen. 
   
   \begin{definitionl}{Konfiguration}{konfig}
    Ein planarer Graph $C$ heißt \textit{Konfiguration}, wenn
    \begin{itemize}
     \item er regulär ist,
     \item die Außenknoten einen Ring der \textit{Ringgröße} $k \geq 4$ bilden,
     \item innere Knoten existieren,
     \item die beschränkten Gebiete von Dreiecken begrenzt werden,
     \item jedes Dreieck Grenze eines Gebiets ist.
    \end{itemize}
   \end{definitionl}
   
  Um eine bessere Vorstellung für diese Graphen zu bekommen, betrachten wir zunächst einige Beispiele. Ein nicht-triviales Beispiel für eine Konfiguration ist der \textit{Birkhoff}-Diamant mit insgesamt 10 Knoten (linkes Bild).
   
  \begin{figure}[hb]
   \label{AHkonfig}
    \[ \begin{tikzpicture}
      \path[shape=circle]
	(0,1) \blacknode(a1){} 
	(1,0) \blacknode(b1){} (1,1) \blacknode(b2){} (1,2) \blacknode(b3){}
	(2,0.5) \blacknode(c1){} (2,1.5) \blacknode(c2){}
	(3,0) \blacknode(d1){} (3,1) \blacknode(d2){} (3,2) \blacknode(d3){}
	(4,1) \blacknode(e1){}
	(7.5,0.5) \blacknode(z1){} (9,0) \blacknode(z2){} (10.5,0.5) \blacknode(z3){} 
	(7.5,1.5) \blacknode(z4){} (9,2) \blacknode(z5){} (10.5,1.5) \blacknode(z6){} 
	(9,1) \blacknode(y){};
	\filldraw (a1) -- (b1) -- (d1) -- (e1) -- (d3) -- (b3) -- (a1) -- (b2);
	\filldraw (b1) -- (b2) -- (b3) -- (c2) -- (d3) -- (d2) -- (d1) -- (c1) -- (b1);
	\filldraw (c1) -- (b2) -- (c2) -- (c1);
	\filldraw (c1) -- (d2) -- (c2);
	\filldraw (d2) -- (e1);
	\filldraw (z1) -- (y) -- (z4);
	\filldraw (z2) -- (y) -- (z5);
	\filldraw (z3) -- (y) -- (z6);
    \end{tikzpicture}  \]
    \caption[Zwei Konfigurationen nach Appel \& Haken: der Birkhoff-Diamant und ein $6$-Stern]{Zwei Konfigurationen nach Appel \& Haken: der Birkhoff-Diamant und ein $6$-Stern}
  \end{figure}
 
   
  Andere Beispiele für Konfigurationen sind \textit{Sterne}. Sie besitzen genau einen inneren Punkt (``\textit{Zentrum}'') und einen Ring von äußeren Punkten, die alle mit dem Zentrum durch eine Kante verbunden sind. Ein Stern heißt $k$-Stern, wenn er genau $k$ äußere Knoten besitzt. Einen $6$-Stern findet man im rechten Bild.
  
  \begin{definition}{Äquivalente Konfigurationen}
   Zwei Konfigurationen $C'=(V',E')$ und $C''=(V'',E'')$ heißen \textit{äquivalent}, wenn es eine Bijektion $\varphi : V' \rightarrow V''$ gibt, die in beide Richtungen die Adjazenzstruktur erhält.
  \end{definition}
  
  Nun können wir davon sprechen, dass ein Graph eine Konfiguration enthält, indem wir folgende Definition bemühen:
  
  \begin{definition}{enthaltene Konfiguration}
   Man sagt, ein Graph $G$ \textit{enthält} eine Konfiguration $C$, wenn es einen geschlossenen Pfad $K$ gibt, sodass der von den Knoten von $K$ und den im Innengebiet liegenden Knoten von $K$ Untergraph $C_K$ von $G$ eine zu $C$ äquivalente Konfiguration ist.
  \end{definition}


  \end{section}
\newpage
 \end{chapter}